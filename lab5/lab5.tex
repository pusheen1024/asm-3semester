\documentclass[bachelor, och, otchet]{../SCWorks}
% Тип обучения (одно из значений):
%    bachelor   - бакалавриат (по умолчанию)
%    spec       - специальность
%    master     - магистратура
% Форма обучения (одно из значений):
%    och        - очное (по умолчанию)
%    zaoch      - заочное
% Тип работы (одно из значений):
%    coursework - курсовая работа (по умолчанию)
%    referat    - реферат
%  * otchet     - универсальный отчет
%  * nirjournal - журнал НИР
%  * digital    - итоговая работа для цифровой кафедры
%    diploma    - дипломная работа
%    pract      - отчет о научно-исследовательской работе
%    autoref    - автореферат выпускной работы
%    assignment - задание на выпускную квалификационную работу
%    review     - отзыв руководителя
%    critique   - рецензия на выпускную работу

% * Добавлены вручную. За вопросами к @mchernigin
\usepackage{../preamble}

\begin{document}

% Кафедра (в родительном падеже)
\chair{информатики и программирования}

% Тема работы
\title{Машинно"=зависимые языки программмирования. Лабораторная работа №5}

% Курс
\course{2}

% Группа
\group{251}

% Факультет (в родительном падеже) (по умолчанию "факультета КНиИТ")
\department{факультета компьютерных наук и информационных технологий}

% Специальность/направление код - наименование
% \napravlenie{02.03.02 "--- Фундаментальная информатика и информационные технологии}
% \napravlenie{02.03.01 "--- Математическое обеспечение и администрирование информационных систем}
% \napravlenie{09.03.01 "--- Информатика и вычислительная техника}
\napravlenie{09.03.04 "--- Программная инженерия}
% \napravlenie{10.05.01 "--- Компьютерная безопасность}

% Для студентки. Для работы студента следующая команда не нужна.
\studenttitle{студентки}

% Фамилия, имя, отчество в родительном падеже
\author{Потапкиной Маргариты Андреевны}

% Заведующий кафедрой 
\chtitle{доцент, к.\,ф.-м.\,н.}
\chname{С.\,В.\,Миронов}

% Руководитель ДПП ПП для цифровой кафедры (перекрывает заведующего кафедры)
% \chpretitle{
%     заведующий кафедрой математических основ информатики и олимпиадного\\
%     программирования на базе МАОУ <<Ф"=Т лицей №1>>
% }
% \chtitle{г. Саратов, к.\,ф.-м.\,н., доцент}
% \chname{Кондратова\, Ю.\,Н.}

% Научный руководитель (для реферата преподаватель проверяющий работу)
\satitle{старший преподаватель} %должность, степень, звание
\saname{Е.\,М.\,Черноусова}

% Руководитель практики от организации (руководитель для цифровой кафедры)
\patitle{доцент, к.\,ф.-м.\,н.}
\paname{С.\,В.\,Миронов}

% Руководитель НИР
\nirtitle{доцент, к.\,п.\,н.} % степень, звание
\nirname{В.\,А.\,Векслер}

% Семестр (только для практики, для остальных типов работ не используется)
\term{2}

% Наименование практики (только для практики, для остальных типов работ не
% используется)
\practtype{учебная}

% Продолжительность практики (количество недель) (только для практики, для
% остальных типов работ не используется)
\duration{2}

% Даты начала и окончания практики (только для практики, для остальных типов
% работ не используется)
\practStart{01.07.2024}
\practFinish{13.01.2024}

% Год выполнения отчета
\date{2025}

\maketitle

% Включение нумерации рисунков, формул и таблиц по разделам (по умолчанию -
% нумерация сквозная) (допускается оба вида нумерации)
\secNumbering

\tableofcontents

% Раздел "Обозначения и сокращения". Может отсутствовать в работе
% \abbreviations
% \begin{description}
%     \item ... "--- ...
%     \item ... "--- ...
% \end{description}

% Раздел "Определения". Может отсутствовать в работе
% \definitions

% Раздел "Определения, обозначения и сокращения". Может отсутствовать в работе.
% Если присутствует, то заменяет собой разделы "Обозначения и сокращения" и
% "Определения"
% \defabbr

\section{Текст задания}
\input{task.txt}

\section{Скриншот запуска программы}
\begin{figure}[H]
	\centering
	\includegraphics{1.png}
\end{figure}

\section{Алгоритм программы}
\begin{enumerate}
	\item Определить в сегменте данных переменные row (номер строки), col (номер столбца), mode (для сохранения режима).
	\item Определить процедуру set\_mode, которая устанавливает новый режим с помощью BIOS"=функции 10h, а также процедуру ret\_mode, которая после выполнения программы вернёт нас к исходному видеорежиму.
	\item Определить процедуру cursor для установки курсора в нужную позицию.
	\item Определить продедуру clear для прокрутки всего экрана вверх (его очистки).
	\item Определить процедуру print для вывода ASCII"=символа.
	\item Вызвать процедуры clear и set\_mode.
	\item Записать в AL ASCII"=код первого символа (41h "--- A), а в BL занести цвет первого ряда символов (0Ah "--- зелёный). Инициализовать переменную row значением 5.
	\item Организовать внешний цикл с меткой rows. Условие завершения цикла "--- когда row станет равен 10, т.е. были выведены 5 рядов. Внутри внешнего цикла увеличивать на 1 row, AL и BL для изменения номера ряда, символа и цвета. Помимо этого каждую итерацию внешнего цикла выставлять переменную col равной 15.
	\item Организовать внутренний цикл с меткой cols. Условие завершения цикла "--- когда col станет равен 25, т.е. были выведены $25-15=10$ рядов. Внутри вложенного цикла вызываем процедуру cursor для того, чтобы изменить положение курсора, процедуру print для печати символа, а также увеличиваем col на 1.
	\item Выполнить запрос на ввод символа с клавиатуры, а затем вызвать процедуру ret\_mode, чтобы программа перешла в исходный режим и завершилась только по нажатию клавиши.
\end{enumerate}

\section{Текст программы на языке ассемблера с комментариями}
\small
\inputminted{nasm}{1.asm}
\normalsize

\section{Ответы на контрольные вопросы}
\begin{enumerate}
\item Каков адрес области видеоданных для:

\begin{itemize}
	\item режимов 00h--06h: B800;
	\item монохромного текстового режима: B000.
\end{itemize}

\item Укажите число страниц, разрешение и число цветов для видеорежима 03.
\begin{itemize}
	\item Число страниц: 4 (0--3).
	\item Разрешение: 720x400.
	\item Число цветов: 16.
\end{itemize}

\item Укажите в двоичной форме содержимое байтов атрибутов для:
\begin{itemize}
	\item пурпурных символов на голубом фоне: 00110101;
	\item зелёных символов на белом мигающем фоне: 11110010.
\end{itemize}

\item Объясните, как ограничивается количество доступных цветов для символа и для фона структурой байта атрибутов.

Старшие 4 бита байта атрибутов отвечают за цвет фона, а младшие 4 бита "--- за цвет символов. Существует $2^4=16$ цветов символов: чёрный, синий, зелёный, голубой, красный, пурпурный, коричневый, белый, серый, светло"=синий, светло"=зелёный, светло"=голубой, светло"=красный, светло"=пурпурный, жёлтый и ярко"=белый (их шестнадцатиричные коды "--- цифры от 0 до F). Что касается фона, то здесь младшие 3 бита отвечают за компоненты R, G и B, каждая из них может быть включена или выключена, старший же бит отвечает за мерцание, поэтому у фона существует только $2^3=8$ цветов. У символа аналогично: 3 младших бита отвечают за R, G, B, но старший бит регулирует яркость (I "--- intensity: может быть нормальная и повышенная). Таким образом, структуру байта можно представить так:
\begin{tabular}{|c|c|c|c|c|c|c|c|}
	\hline
	\multicolumn{4}{|c|}{Фон} & \multicolumn{4}{|c|}{Символ} \\
	\hline
	BL & R & G & B & I & R & G & B \\
	\hline
\end{tabular}

\item Укажите инструкции, необходимые для вывода на экран с помощью функции 09h прерывания INT 10h:
\begin{itemize}
	\item 10 жёлтых сердечек (ASCII 03h) на синем фоне:
	\begin{minted}{nasm}
mov AH,09h
mov AL,03h
mov BH,0
mov BL,00011110b
mov CX,10
int 10h
	\end{minted}
	
	\item 5 белых звёздочек (ASCII 2Ah) на красном фоне.
	\begin{minted}{nasm}
mov AH,09h
mov AL,2Ah
mov BH,0
mov BL,01000111b
mov CX,5
int 10h
	\end{minted}
\end{itemize}

\begin{figure}[H]
	\centering
	\includegraphics{ex1.png}
	\caption{Пример 1}
	\includegraphics{ex2.png}
	\caption{Пример 2}
\end{figure}

\end{enumerate}
% Отобразить все источники. Даже те, на которые нет ссылок.
% \nocite{*}

% Окончание основного документа и начало приложений Каждая последующая секция
% документа будет являться приложением
\appendix
\end{document}

